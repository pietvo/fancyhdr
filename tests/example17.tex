\documentclass[openany]{book}
\usepackage{lipsum}
\usepackage{extramarks}
 \usepackage{fancyhdr}
 \pagestyle{fancy}
 \fancyhead[L]{\firstrightxmark}
 \fancyfoot[R]{\lastleftxmark}
\fancypagestyle{plain}{\fancyhead{}\renewcommand{\headrule}{}}

\begin{document}

\pagenumbering{roman}
\tableofcontents

\bigskip

This is a test for the \texttt{extramarks} package to implement a ``Continued\ldots'' header/footer. This document uses a better solution. But I am still not completely confident that it covers all cases. More investigation is necessary.

\newpage
\pagenumbering{arabic}
\chapter{Introduction}

\lipsum[1]

\section{The Problem}
\label{sec:problem}

\noindent\rule{\textwidth}{1mm}\\
\extramarks {Continued on next page\ldots}{Continued\ldots}
\textbf{We want to indicate that this block of text belongs together, with `Continued' in header and footer.}

\lipsum[2]
\lipsum

\textbf{Here ends the  block of text that belongs together.}\\
\noindent\rule{\textwidth}{1mm}
\extramarks{}{Continued\ldots}
\extramarks{}{}

\section{Evaluation}

\lipsum[3-9]

\chapter{Another chapter}

\label{cha:another-chapter}

\lipsum[2]

\section{Another section}

\lipsum[3-4]

\noindent\rule{\textwidth}{1mm}\\
\fancyhead[L]{Continued}
\fancyfoot[R]{Continued on next page\ldots}
\textbf{Here we use the wrong way by directly manipulating the headers/footers without using marks. You can see that it doesn't work well.}

\lipsum[5]

\textbf{End of block.}\\
\noindent\rule{\textwidth}{1mm}\\
\fancyhead[L]{}
\fancyfoot[R]{}

\end{document}
